\documentclass{beamer}
\usepackage[utf8]{inputenc}
\usepackage[T1]{fontenc}
\usepackage[brazil]{babel}
\usepackage{amsmath,amsfonts,amssymb,amsthm}
\usepackage{proof}
\usepackage{color}
 
\usetheme{Luebeck}
 
\title{Introdução ao $\lambda$-Cálculo Atipado}
\subtitle{Teoria de Tipos}
\author[Prof. Rodrigo Ribeiro]{Prof. Rodrigo Ribeiro}
\institute{Departamento de Computação e Sistemas}
\date{\today}

%% ODER: format ==         = "\mathrel{==}"
%% ODER: format /=         = "\neq "
%
%
\makeatletter
\@ifundefined{lhs2tex.lhs2tex.sty.read}%
  {\@namedef{lhs2tex.lhs2tex.sty.read}{}%
   \newcommand\SkipToFmtEnd{}%
   \newcommand\EndFmtInput{}%
   \long\def\SkipToFmtEnd#1\EndFmtInput{}%
  }\SkipToFmtEnd

\newcommand\ReadOnlyOnce[1]{\@ifundefined{#1}{\@namedef{#1}{}}\SkipToFmtEnd}
\usepackage{amstext}
\usepackage{amssymb}
\usepackage{stmaryrd}
\DeclareFontFamily{OT1}{cmtex}{}
\DeclareFontShape{OT1}{cmtex}{m}{n}
  {<5><6><7><8>cmtex8
   <9>cmtex9
   <10><10.95><12><14.4><17.28><20.74><24.88>cmtex10}{}
\DeclareFontShape{OT1}{cmtex}{m}{it}
  {<-> ssub * cmtt/m/it}{}
\newcommand{\texfamily}{\fontfamily{cmtex}\selectfont}
\DeclareFontShape{OT1}{cmtt}{bx}{n}
  {<5><6><7><8>cmtt8
   <9>cmbtt9
   <10><10.95><12><14.4><17.28><20.74><24.88>cmbtt10}{}
\DeclareFontShape{OT1}{cmtex}{bx}{n}
  {<-> ssub * cmtt/bx/n}{}
\newcommand{\tex}[1]{\text{\texfamily#1}}	% NEU

\newcommand{\Sp}{\hskip.33334em\relax}


\newcommand{\Conid}[1]{\mathit{#1}}
\newcommand{\Varid}[1]{\mathit{#1}}
\newcommand{\anonymous}{\kern0.06em \vbox{\hrule\@width.5em}}
\newcommand{\plus}{\mathbin{+\!\!\!+}}
\newcommand{\bind}{\mathbin{>\!\!\!>\mkern-6.7mu=}}
\newcommand{\rbind}{\mathbin{=\mkern-6.7mu<\!\!\!<}}% suggested by Neil Mitchell
\newcommand{\sequ}{\mathbin{>\!\!\!>}}
\renewcommand{\leq}{\leqslant}
\renewcommand{\geq}{\geqslant}
\usepackage{polytable}

%mathindent has to be defined
\@ifundefined{mathindent}%
  {\newdimen\mathindent\mathindent\leftmargini}%
  {}%

\def\resethooks{%
  \global\let\SaveRestoreHook\empty
  \global\let\ColumnHook\empty}
\newcommand*{\savecolumns}[1][default]%
  {\g@addto@macro\SaveRestoreHook{\savecolumns[#1]}}
\newcommand*{\restorecolumns}[1][default]%
  {\g@addto@macro\SaveRestoreHook{\restorecolumns[#1]}}
\newcommand*{\aligncolumn}[2]%
  {\g@addto@macro\ColumnHook{\column{#1}{#2}}}

\resethooks

\newcommand{\onelinecommentchars}{\quad-{}- }
\newcommand{\commentbeginchars}{\enskip\{-}
\newcommand{\commentendchars}{-\}\enskip}

\newcommand{\visiblecomments}{%
  \let\onelinecomment=\onelinecommentchars
  \let\commentbegin=\commentbeginchars
  \let\commentend=\commentendchars}

\newcommand{\invisiblecomments}{%
  \let\onelinecomment=\empty
  \let\commentbegin=\empty
  \let\commentend=\empty}

\visiblecomments

\newlength{\blanklineskip}
\setlength{\blanklineskip}{0.66084ex}

\newcommand{\hsindent}[1]{\quad}% default is fixed indentation
\let\hspre\empty
\let\hspost\empty
\newcommand{\NB}{\textbf{NB}}
\newcommand{\Todo}[1]{$\langle$\textbf{To do:}~#1$\rangle$}

\EndFmtInput
\makeatother
%
%
%
%
%
%
% This package provides two environments suitable to take the place
% of hscode, called "plainhscode" and "arrayhscode". 
%
% The plain environment surrounds each code block by vertical space,
% and it uses \abovedisplayskip and \belowdisplayskip to get spacing
% similar to formulas. Note that if these dimensions are changed,
% the spacing around displayed math formulas changes as well.
% All code is indented using \leftskip.
%
% Changed 19.08.2004 to reflect changes in colorcode. Should work with
% CodeGroup.sty.
%
\ReadOnlyOnce{polycode.fmt}%
\makeatletter

\newcommand{\hsnewpar}[1]%
  {{\parskip=0pt\parindent=0pt\par\vskip #1\noindent}}

% can be used, for instance, to redefine the code size, by setting the
% command to \small or something alike
\newcommand{\hscodestyle}{}

% The command \sethscode can be used to switch the code formatting
% behaviour by mapping the hscode environment in the subst directive
% to a new LaTeX environment.

\newcommand{\sethscode}[1]%
  {\expandafter\let\expandafter\hscode\csname #1\endcsname
   \expandafter\let\expandafter\endhscode\csname end#1\endcsname}

% "compatibility" mode restores the non-polycode.fmt layout.

\newenvironment{compathscode}%
  {\par\noindent
   \advance\leftskip\mathindent
   \hscodestyle
   \let\\=\@normalcr
   \let\hspre\(\let\hspost\)%
   \pboxed}%
  {\endpboxed\)%
   \par\noindent
   \ignorespacesafterend}

\newcommand{\compaths}{\sethscode{compathscode}}

% "plain" mode is the proposed default.
% It should now work with \centering.
% This required some changes. The old version
% is still available for reference as oldplainhscode.

\newenvironment{plainhscode}%
  {\hsnewpar\abovedisplayskip
   \advance\leftskip\mathindent
   \hscodestyle
   \let\hspre\(\let\hspost\)%
   \pboxed}%
  {\endpboxed%
   \hsnewpar\belowdisplayskip
   \ignorespacesafterend}

\newenvironment{oldplainhscode}%
  {\hsnewpar\abovedisplayskip
   \advance\leftskip\mathindent
   \hscodestyle
   \let\\=\@normalcr
   \(\pboxed}%
  {\endpboxed\)%
   \hsnewpar\belowdisplayskip
   \ignorespacesafterend}

% Here, we make plainhscode the default environment.

\newcommand{\plainhs}{\sethscode{plainhscode}}
\newcommand{\oldplainhs}{\sethscode{oldplainhscode}}
\plainhs

% The arrayhscode is like plain, but makes use of polytable's
% parray environment which disallows page breaks in code blocks.

\newenvironment{arrayhscode}%
  {\hsnewpar\abovedisplayskip
   \advance\leftskip\mathindent
   \hscodestyle
   \let\\=\@normalcr
   \(\parray}%
  {\endparray\)%
   \hsnewpar\belowdisplayskip
   \ignorespacesafterend}

\newcommand{\arrayhs}{\sethscode{arrayhscode}}

% The mathhscode environment also makes use of polytable's parray 
% environment. It is supposed to be used only inside math mode 
% (I used it to typeset the type rules in my thesis).

\newenvironment{mathhscode}%
  {\parray}{\endparray}

\newcommand{\mathhs}{\sethscode{mathhscode}}

% texths is similar to mathhs, but works in text mode.

\newenvironment{texthscode}%
  {\(\parray}{\endparray\)}

\newcommand{\texths}{\sethscode{texthscode}}

% The framed environment places code in a framed box.

\def\codeframewidth{\arrayrulewidth}
\RequirePackage{calc}

\newenvironment{framedhscode}%
  {\parskip=\abovedisplayskip\par\noindent
   \hscodestyle
   \arrayrulewidth=\codeframewidth
   \tabular{@{}|p{\linewidth-2\arraycolsep-2\arrayrulewidth-2pt}|@{}}%
   \hline\framedhslinecorrect\\{-1.5ex}%
   \let\endoflinesave=\\
   \let\\=\@normalcr
   \(\pboxed}%
  {\endpboxed\)%
   \framedhslinecorrect\endoflinesave{.5ex}\hline
   \endtabular
   \parskip=\belowdisplayskip\par\noindent
   \ignorespacesafterend}

\newcommand{\framedhslinecorrect}[2]%
  {#1[#2]}

\newcommand{\framedhs}{\sethscode{framedhscode}}

% The inlinehscode environment is an experimental environment
% that can be used to typeset displayed code inline.

\newenvironment{inlinehscode}%
  {\(\def\column##1##2{}%
   \let\>\undefined\let\<\undefined\let\\\undefined
   \newcommand\>[1][]{}\newcommand\<[1][]{}\newcommand\\[1][]{}%
   \def\fromto##1##2##3{##3}%
   \def\nextline{}}{\) }%

\newcommand{\inlinehs}{\sethscode{inlinehscode}}

% The joincode environment is a separate environment that
% can be used to surround and thereby connect multiple code
% blocks.

\newenvironment{joincode}%
  {\let\orighscode=\hscode
   \let\origendhscode=\endhscode
   \def\endhscode{\def\hscode{\endgroup\def\@currenvir{hscode}\\}\begingroup}
   %\let\SaveRestoreHook=\empty
   %\let\ColumnHook=\empty
   %\let\resethooks=\empty
   \orighscode\def\hscode{\endgroup\def\@currenvir{hscode}}}%
  {\origendhscode
   \global\let\hscode=\orighscode
   \global\let\endhscode=\origendhscode}%

\makeatother
\EndFmtInput
%

\DeclareMathAlphabet{\mathkw}{OT1}{cmss}{bx}{n}


\newcommand{\redFG}[1]{\textcolor[rgb]{0.6,0,0}{#1}}
\newcommand{\greenFG}[1]{\textcolor[rgb]{0,0.4,0}{#1}}
\newcommand{\blueFG}[1]{\textcolor[rgb]{0,0,0.8}{#1}}
\newcommand{\orangeFG}[1]{\textcolor[rgb]{0.8,0.4,0}{#1}}
\newcommand{\purpleFG}[1]{\textcolor[rgb]{0.4,0,0.4}{#1}}
\newcommand{\yellowFG}[1]{\textcolor{yellow}{#1}}
\newcommand{\brownFG}[1]{\textcolor[rgb]{0.5,0.2,0.2}{#1}}
\newcommand{\blackFG}[1]{\textcolor[rgb]{0,0,0}{#1}}
\newcommand{\whiteFG}[1]{\textcolor[rgb]{1,1,1}{#1}}
\newcommand{\yellowBG}[1]{\colorbox[rgb]{1,1,0.2}{#1}}
\newcommand{\brownBG}[1]{\colorbox[rgb]{1.0,0.7,0.4}{#1}}

\newcommand{\ColourStuff}{
  \newcommand{\red}{\redFG}
  \newcommand{\green}{\greenFG}
  \newcommand{\blue}{\blueFG}
  \newcommand{\orange}{\orangeFG}
  \newcommand{\purple}{\purpleFG}
  \newcommand{\yellow}{\yellowFG}
  \newcommand{\brown}{\brownFG}
  \newcommand{\black}{\blackFG}
  \newcommand{\white}{\whiteFG}
}

\ColourStuff

\newcommand{\D}[1]{\blue{\mathsf{#1}}}
\newcommand{\C}[1]{\green{\mathsf{#1}}}
\newcommand{\F}[1]{\green{\mathsf{#1}}}
\newcommand{\V}[1]{\blue{\mathit{#1}}}
\newcommand{\N}[1]{\purple{\mathit{#1}}}
\newcommand{\K}[1]{\red{\mathkw{#1}}}

 
\begin{document}
   \begin{frame}
       \titlepage
   \end{frame}
   \begin{frame}{$\lambda$-Cálculo Atipado --- (I)}
      \begin{block}{O $\lambda$-cálculo}
         \begin{itemize}
            \item Sistema formal capaz de representar qualquer função computável.
            \item Pode ser vista como linguagem de programação mínima.
            \item Inicialmente concebida para uso em lógica.
         \end{itemize}
      \end{block}
   \end{frame}
   \begin{frame}{$\lambda$-Cálculo Atipado --- (II)}
      \begin{block}{Sintaxe}
         \[
             \begin{array}{rcll}
                t & ::= & x & \text{variável} \\
                  &     & \lambda x. t & \text{abstração}\\
                  &     & t\:t & \text{aplicação}
             \end{array}
         \]
      \end{block}
   \end{frame}
   \begin{frame}{$\lambda$-Cálculo Atipado --- (III)}
      \begin{block}{Escopo}
         \begin{itemize}
            \item Variável ligada: $x$ é ligada se ocorre no termo $t$ em $\lambda x.t$.
            \item Variável livre: $x$ é livre se não ocorre no escopo de uma abstração envolvendo $x$.
         \end{itemize}
      \end{block}
   \end{frame}
   \begin{frame}{$\lambda$-Cálculo Atipado --- (IV)}
      \begin{block}{Semântica Operacional}
         \begin{itemize}
            \item Elemento fundamental: substituição.
            \[
                 (\lambda x.t_1)\: t_2 \to [x \mapsto t_2]\:t_1
            \]
         \end{itemize}
      \end{block}
   \end{frame}
   \begin{frame}{$\lambda$-Cálculo Atipado --- (V)}
      \begin{block}{Representando Booleanos}
         \[
             \begin{array}{lcl}
                true & = & \lambda t. \lambda f. t\\
                false & = & \lambda t. \lambda f. f\\
             \end{array}
         \]
      \end{block}
   \end{frame}
   \begin{frame}{$\lambda$-Cálculo Atipado --- (VI)}
      \begin{block}{Funções sobre Booleanos}
         \[
             \begin{array}{lcl}
                not & = & \lambda x . x\: false\: true
             \end{array}
         \]
      \end{block}
   \end{frame}
   \begin{frame}{$\lambda$-Cálculo Atipado --- (VII)}
      \begin{block}{Exemplo}
         \[
             \begin{array}{lc}
                not\:\: true & = \\
                (\lambda x .\, x\: false\: true)\: true & = \\
                true\:\: false\:\: true & = \\
                (\lambda t. \lambda f. t)\: false\: true & = \\
                \lambda f. false & = \\
                false & = \\
                \lambda t. \lambda f. f
             \end{array}
         \]
      \end{block}
   \end{frame}
   \begin{frame}{$\lambda$-Cálculo Atipado --- (VIII)}
      \begin{block}{Representando números}
         \[
             \begin{array}{lcl}
                0 & = & \lambda s. \lambda z. z \\
                1 & = & \lambda s. \lambda z. s\: z \\
                  & ... & \\
                suc & = & \lambda n. \lambda s. \lambda z. s (n\:\:s\:\:z)
             \end{array}
         \]
      \end{block}
   \end{frame}
   \begin{frame}{$\lambda$-Cálculo Atipado --- (IX)}
      \begin{block}{Operações sobre números}
         \[
             \begin{array}{lcl}
                plus & = & \lambda m.\lambda n.\lambda n.\lambda z. m\:\:s\:\:(n\:\:s\:\:z)\\
                times & = & \lambda m.\lambda n. m\:\:(plus\:\:n)\:\:0\\
             \end{array}
         \]
      \end{block}
   \end{frame}
   \begin{frame}{$\lambda$-Cálculo Atipado --- (X)}
      \begin{block}{Formalidades}
         \begin{itemize}
            \item Cálculo de variáveis livres
            \[
                \begin{array}{lcl}
                   fv(x) & = & \{x\}\\
                   fv(\lambda x.t_1) & = & fv(t_1) - \{x\}\\
                   fv(t_1\:\:t_2) & = & fv(t_1) \cup fv(t_2)\\
                \end{array}
            \]
         \end{itemize}
      \end{block}
   \end{frame}
   \begin{frame}{$\lambda$-Cálculo Atipado --- (XI)}
      \begin{block}{Substituição}
           \[
               \begin{array}{lcll}
                  {[x \mapsto s]}\:x               & = & s                              & \\
                  {[x \mapsto s]}\:y               & = & y                              & x \neq y\\
                  {[x \mapsto s]}\:(\lambda y.t_1) & = & \lambda y. {[x \mapsto s]}\: t_1 & y \neq x \land y \not\in fv(s)\\
                  {[x \mapsto s]}\:(t_1\:\:t_2)    & = & ({[x\mapsto s]}\:t_1)({[x\mapsto s]}\:t_2)\\
               \end{array}
           \]
      \end{block}
   \end{frame}
   \begin{frame}{$\lambda$-Cálculo --- (XII)}
      \begin{block}{Alguns conceitos...}
         \begin{itemize}
            \item Captura de variável.
            \item Termos fechados, termos $\alpha$-equivalentes.
            \item Valores: $\lambda$-abstrações.
         \end{itemize}
      \end{block}
   \end{frame}
   \begin{frame}{$\lambda$-Cálculo --- (XIII)}
      \begin{block}{Semântica Operacional}
          \[
              \begin{array}{c}
                  \fbox{$t\to t'$}\\ \\
                  \begin{array}{c}
                     \infer[_{(EAppAbs)}]
                           {\lambda x. t_{12}\:v_2 \to [x\mapsto v_2]\:t_{12}}
                           {} \\ \\
                     \infer[_{(EApp_1)}]
                           {t_1\:\:t_2\to t'_1\:\:t_2}
                           {t_1\to t'_1} \\ \\
                     \infer[_{(EApp_2)}]
                           {v_1\:\:t_2\to v_1\:\:t'_2}
                           {t_2 \to t'_2}
                  \end{array}
              \end{array}
          \]
      \end{block}
   \end{frame}
\end{document}
